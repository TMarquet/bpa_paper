%%%% IACR Transactions TEMPLATE %%%%
% This file shows how to use the iacrtrans class to write a paper.
% Written by Gaetan Leurent gaetan.leurent@inria.fr (2020)
% Public Domain (CC0)


%%%% 1. DOCUMENTCLASS %%%%
\documentclass[journal=tosc,submission]{iacrtrans}
%%%% NOTES:
% - Change "journal=tosc" to "journal=tches" if needed
% - Change "submission" to "final" for final version
% - Add "spthm" for LNCS-like theorems


%%%% 2. PACKAGES %%%%
\usepackage{lipsum} % Example package -- can be removed


%%%% 3. AUTHOR, INSTITUTE %%%%
\author{Jane Doe\inst{1,2} \and John Doe\inst{1}}
\institute{
  Institute A, City, Country, \email{jane@institute}
  \and
  Institute B, City, Country, \email{john@institute}
}
%%%% NOTES:
% - We need a city name for indexation purpose, even if it is redundant
%   (eg: University of Atlantis, Atlantis, Atlantis)
% - \inst{} can be omitted if there is a single institute,
%   or exactly one institute per author


%%%% 4. TITLE %%%%
\title{My New Paper}
%%%% NOTES:
% - If the title is too long, or includes special macro, please
%   provide a "running title" as optional argument: \title[Short]{Long}
% - You can provide an optional subtitle with \subtitle.

\begin{document}

\maketitle


%%%% 5. KEYWORDS %%%%
\keywords{Deep Learning \and Side Channels Attacks \and Data Augmentation \and AES}


%%%% 6. ABSTRACT %%%%
\begin{abstract}
  Deep Learning techniques have been proven to be very efficient in the context of side channels attacks and are extensively studied in the recent literature especially in cases of single target DPA attacks. Most of those studies focusing on one byte of one specific intermediate which is most of the time the SubBytes output. Those approachs can naturally be extended to other intermediates to perform multi-target attacks but it require to adapt the model's hyperparameters again and is therefore extremely time consuming. The purpose of this work is to propose an efficient way to consistently achieve good results on all intermediates with the minimal amount of retraining. To do so, we propose that all bytes of one intermediate can be mixed together in order to perform a kind of data augmentation.
\end{abstract}


%%%% 7. PAPER CONTENT %%%%
\section{Introduction}

Widely used primitives like the AES~\cite{AES} do not have perfect
security, and can be analysed with linear
cryptanalysis~\cite{EC:Matsui93}, differential
cryptanalysis~\cite{JC:BihSha91}, or differential power
analysis~\cite{C:KocJafJun99}.  We show that the One-Time-Pad is
unconditionally secure in \autoref{sec:main}.

\lipsum[9]

\section{Main Result}
\label{sec:main}

\lipsum


%%%% 8. BILBIOGRAPHY %%%%
\bibliographystyle{alpha}
\bibliography{abbrev3,crypto,biblio}
%%%% NOTES
% - Download abbrev3.bib and crypto.bib from https://cryptobib.di.ens.fr/
% - Use bilbio.bib for additional references not in the cryptobib database.
%   If possible, take them from DBLP.

\end{document}
